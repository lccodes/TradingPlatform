\documentclass[letterpaper,11pt]{article}

% Encoding
\usepackage[utf8]{inputenc}

% Page structure
\usepackage{geometry}
\geometry{margin=2.54cm}
\usepackage[skins]{tcolorbox}
%\geometry{top=1.54cm}

% Titles
\usepackage{sectsty}
\allsectionsfont{\sffamily\mdseries\upshape}

% Headers and Footers
\usepackage{fancyhdr}
\pagestyle{fancy}

% Graphics, figures and plots
\usepackage{graphicx}
\usepackage{float}
\usepackage{pgfplots}
\usepackage{caption}
\usepackage{subcaption}
\usepackage{changepage}

% Etc
\usepackage{hyperref}

% Bibliography
\usepackage[numbers]{natbib}

%%%%%%%%%%
% Math
\usepackage{amsmath}
\usepackage{amsthm}
\usepackage{amsfonts}
\usepackage{amssymb}
\usepackage{amscd}
\usepackage{eurosym}
\usepackage{mathrsfs}
\usepackage{mathtools}
\usepackage{bm} % defines a com­mand \bm which makes its ar­gu­ment bold
\usepackage{bbm}

\usepackage{enumitem, hyperref}
\makeatletter
\def\namedlabel#1#2{\begingroup
    #2%
    \def\@currentlabel{#2}%
    \phantomsection\label{#1}\endgroup
}
\makeatother

%%%% Mathematical Environments
\theoremstyle{plain}
\newtheorem{theorem}{Theorem}[section]
\newtheorem{lemma}[theorem]{Lemma}
\newtheorem{corollary}[theorem]{Corollary}
\newtheorem{proposition}[theorem]{Proposition}
\newtheorem{conjecture}[theorem]{Conjecture}

\theoremstyle{definition}
\newtheorem{definition}[theorem]{Definition}
\newtheorem{example}[theorem]{Example}

\theoremstyle{remark}
\newtheorem{remark}[theorem]{Remark}
\newtheorem{note}[theorem]{Note}
\newtheorem{case}[theorem]{Case}

%%%%%%%%%%
% Pseudocode
\usepackage[ruled]{algorithm2e}
%\usepackage{algpseudocode}
%\usepackage{verbatim}

%%%%%%%%%%
% Operators
\DeclareMathOperator*{\argmax}{arg\,max}
\DeclareMathOperator*{\argmin}{arg\,min}
\DeclareMathOperator*{\Exp}{\mathbb{E}}
\DeclareMathOperator*{\Var}{\text{Var}}
\DeclareMathOperator*{\Cov}{\text{Cov}}

%%%%%%%%%%
% Paired -- \command* will automatically resize
\DeclarePairedDelimiter\paren{(}{)}           % (parentheses)
\DeclarePairedDelimiter\ang{\langle}{\rangle} % <angle brackets>
\DeclarePairedDelimiter\abs{\lvert}{\rvert}   % |absolute value|
\DeclarePairedDelimiter\norm{\lVert}{\rVert}  % ||norm||
\DeclarePairedDelimiter\bkt{[}{]}             % [brackets]
\DeclarePairedDelimiter\set{\{}{\}}           % {braces}


%%%%%%%%%%

\title{Market Maker Experiments}

\author{Luke Camery and Amy Greenwald}

%%%%