\section{Motivation}

So... LMSR... while great in theory (why?? path independence), sucks in practice (why??).
We want to 

\section{Goals}
To test five different market maker mechanisms for their liquidity sensitivity,
profit expectation and accuracy.  


\section{Definitions}

We denote time by $\Time\in\mathbb{R}_+$. We denote money by $\Money\in\mathbb{R}$.\\

An \mydef{event} has an outcome where we restrict our attention to a binary outcome. 
The outcome is equal to YES in case the event occurs and NO otherwise. We assume there 
is a way to unambiguously determine the outcome of an event. The mechanism for making
the determination is $\R$.\\

$\R$ is an oracle for mapping an \mydef{event} to an outcome.\\

An \mydef{option} is a security that yields a return depending on the outcome of an 
event. Each \mydef{option} has a price, a $\Option_{\Time}$ when $\R(\Option)$ will be
evaluated, and an outcome. The \mydef{option} will convert to \$1 if $\Option_{outcome}$
equals $\R(\Option)$ otherwise it converts to \$0.\\

The function $\OptionPrice(\Option,\Time)$ at time $\Time$ reflects the likelihood 
of the $\Option_{outcome}$ being realized at the strike date $\Option_{\Time}$.\\

An \mydef{agent} $\Agent\in\Agents$ has a belief $\AgentBelief_{\Agent\Time} \in [0,1]$ 
at time $\Time$ and a budget $\AgentBudget$, which is set a priori and fixed. The $\AgentBelief_{\Agent\Time}$ 
reflects the agent's private belief in the expected likelihood of the $\Option_{outcome}$ 
being realized. This belief maps directly to the price that the agent is willing to pay 
for the $\Option$.\\

A \mydef{prediction market} $\Market$ trades outcomes in an event. Formally, a prediction market 
is a tuple $\left<\Option_0,\Option_1, \Agents, \Book\right>$. Each \mydef{agent} 
$\Agent\in\Agents$ purchases some number of $\Option_0$ and $\Option_1$ paying the price 
quoted by the \mydef{prediction market} at each time $\Time$ when the \mydef{agent} $\Agent$ 
made the purchases. A prediction market has a \mydef{book} $\Book$ that maps $\Agent\in\Agents$ to a tuple $\left<\mathbb{R},\mathbb{R}\right>$ which tracks the number of $\Option_0$ and $\Option_1$ 
purchased by that \mydef{agent} $\Agent$.

A \mydef{book} $\Book_{\Market,\Time}$ accepts an \mydef{agent} $\Agent$ and returns the number of 
each \mydef{option} $\Option$ that \mydef{agent} $\Agent$ has purchased at time $\Time$. A 
\mydef{book} $\Book$ at time $\Time$ records all the transactions made up to time $\Time$.\\

A \mydef{market maker} $\MarketMaker$ is \mydef{predciction market} and strategy for setting prices
on each \mydef{option} $\Option_0$ and $\Option_1$ at time $\Time$. Each \mydef{market maker} has
their own function $\Price(\Option,\Time,\Agent,\mathbb{R})$ where the quantity $\mathbb{R}$ is the
number of the option desired by the \mydef{agent} $\Agent$, and the output is the money $\Money$
required by the \mydef{market maker} $\MarketMaker$ from the \mydef{agent} $\Agent$. \\

We formalize the pricing fnction as$\Price:\left<\Option,\Time,\Agent,\mathbb{R}\right> \rightarrow \Money$.\\

A \mydef{market} is defined as a \mydef{prediction market}.\\

The \mydef{market depth} $\MarketDepth$ is the number of \mydef{agents} that have participated in the market.\\

\section{Market Makers}
In these experiments we will test the following five market markers.
\subsection{Logarithmic Market Scoring Rule}
\subsection{Luke's Online Budget Weighted Average}
\subsection{Yiling and Jen's Expert Weighted Majority}
\subsection{Luke's Weighted Majority}
\subsection{Practical Liquidity Sensitive Market Maker}

\section{Setup}

\subsection{Types}
A \mydef{BasicAgent} follows from the definition of these \mydef{market makers} as truthful
mechanisms. Each \mydef{BasicAgent} reports its belief $\AgentBelief$ and budget $\AgentBudget$
truthfully when requested by the mechanism. Each \mydef{BasicAgent} is rational and profit
maximizing.

\subsection{Metrics}
\subsubsection{Liquidity Sensitivity}
This research uses a novel definition of the vague concept of liquidity sensitivity. The need
for a unified definiton stems from the known issue that LMSR can provide \"too little\" or 
\"too much\" weight to each agent depending on its liquidity parameter and market depth. One extreme
example us that an infinitesimally low liquidity parameter will allow an \mydef{agent} with a nonzero
budget to dictate the market price regardless of how many \mydef{agents} have already participated in
the market.\\

We define a \mydef{market maker} as \mydef{liquidity sensitive} if two \mydef{agents} that have identical
transactions in the \mydef{market} will have equal impacts on the \mydef{market price} proportional to 
the \mydef{market depth} $\MarketDepth_{1}$ at time $\Time_{1}$ of the first purchase, the \mydef{market depth} 
$\MarketDepth_{2}$ at time $\Time_{2}$ of the second purchase, and the difference in time $\Time_{1} - \Time_{2}$.\\

We formalize \mydef{liquidity sensitive} as the function $\Delta = f(\MarketDepth_{1},\MarketDepth_{2},\Time_{\Delta})$
having a constant $\Delta$.\\

\subsubsection{Market Maker Profit}
\subsubsection{Social Welfare}
\subsubsection{Accuracy}
\textbf{Regret}
\textbf{Expectation}
\textbf{Mean Squared Error}
\subsubsection{Precision}

\subsection{Experimental Design}