\section{Model}

\subsection{Definitions}

We denote \mydef{time} by $\Time\in\mathbb{R}_+$. \\

An \mydef{event} $\Event$ has an outcome at time $\Event_\Time \in \mathbb{R}_+$ 
where we restrict our attention to a binary outcome. 
The outcome is equal to YES in case the event occurs, and NO otherwise. 
We denote the YES outcome with a 1, and the NO outcome with a 0.
We assume there is a way to unambiguously determine the outcome of an event. \\

$\Revealer$ is an \mydef{oracle} for mapping an event to an outcome.
We denote the outcome of event $\Event$ under $\Revealer$ as 
$\Revealer(\Event) \in \{0,1\}$.\\

An \mydef{option} $\Option$ is a security that yields a return depending on 
the outcome of an event $\Option_\Event$ at time $\Event_\Time$. Each option has a direction 
$\Option_\OptionDirection \in \{0, 1\}$ and a strike time $\Option_\Time = \Event_\Time$ 
when $\Revealer(\Option_\Event)$ will be evaluated. 
The option will convert to \$1 at time $\Option_\Time$ if $\Revealer(\Option_\Event)$ equals 
$\Option_\OptionDirection$, otherwise it converts to \$0. Two options are said to be
\mydef{complementary options} if they trade opposite directions in the same event. Given an
event $\Event$, we denote the complementary options as 
$\OptionComplements = \left<\Option_{0}, \Option_{1}\right>$.\\

Given complementary options $\Option_{0}$ and $\Option_{1}$,
let $\Agents_{\Option_0}$ be the \mydef{set of agents} that acquire option
$\Option_{0}$, and similarly let $\Agents_{\Option_1}$ be the set 
of agents that acquire option $\Option_{1}$. 
Let $\Agents_\Option = \Agents_{\Option_0} \cup \Agents_{\Option_1}$
be the set of all agents trading complementary options.\\

An \mydef{agent} $\Agent\in\Agents_\Option$ has a \mydef{private belief} 
$\Agent_\AgentBelief \in [0,1]$ 
and a \mydef{budget} $\Agent_\AgentBudget \in \mathbb{R}_+$.
The agent's private belief $\Agent_\AgentBelief$ 
is the subjective probability that the agent
assigns to the outcome of event $\Option_\Event$ being direction $\Option_\OptionDirection$
at strike time $\Option_\Time$.\\

An agent's $\Agent\in\Agents_\Option$ \mydef{strategy} 
$\AgentStrategy_\Agent(\Time) \in \mathbb{R}_+$ specifies the quantity of option 
$\Option$ purchased by the agent at time $\Time$.\\

A \mydef{prediction market} $\Market$ trades complementary options. 
Formally, a prediction market is a tuple $\Market = 
\left<\Option_c, \Agents_\Option \right>$ where each \mydef{agent} 
$\Agent\in\Agents_\Option$ purchases 
either some number of $\Option_0$ or $\Option_1$ options paying the price quoted by the 
market maker.\\


A \mydef{market maker} $\OptionPrice_\Market$ is a function 
that maps an option $\Option$, an agent $\Agent \in \Agents_\Option$ 
and a quantity $\Quantity\in\mathbb{R}_+$ at time $\Time \in [0,\Time_\Option)$ 
to a price $\OptionPrice(\Option,\Agent,\Quantity,\Time) \in \mathbb{R}_+$. 
By definition $\OptionPrice(\Option,\Agent, \Time) = \Revealer(\Option_\Event)$ if 
$\Time \ge \Time_\Option$.

\subsection{Assumptions}

\commente{WITHOUT LOSS OF GENERALITY???}
In our model we assume that agent $\Agent \in \Agents_\Option$ has an arrival time 
$\Agent_\Time \in \mathbb{R}_+$ where $\Agent_\Time < \Option_\Time$, 
and execute their strategy only once at time $\Agent_\Time$.\\

We assume that agents' private beliefs and budgets are not common knowledge
but are drawn from known distributions $\AgentBelief \sim v(\cdot)$ and
$\AgentBudget \sim b(\cdot)$.\\

We assume agents are allowed to observe the true current price of any option.\\

We assume that the market maker mechanism and all its parameters are common knowledge.\\

%The \mydef{market depth} $\MarketDepth$ is the number of \mydef{agents} that have 
%participated in the prediction market, i.e. $|\Agents_\Option|$.

\subsection{Agent Behavior}
\commentl{Define RE and PI here} \\

A major flaw in the current literature, which is reflected in LMSR and the agent behavior theories, is the generalization that agents cannot choose when to enter the market. Although fixing entry time is useful for equilibrium calculations nevertheless it is important to theorize about the value of a certain entry time $\Time$. Under a model based in RE where the collective signal is truthful, a rational agent would want to be the last decision maker and then have the outcome revealed provided that $N > 2$.\\

We present a novel theory called Prior Information Timing (PIT) where agents choose valuations based on a linear combination of their signal and the market price, but value deferring this assesment in order to gain more information. Risk neutral agents can price the ability to defer the decision until time $\Time + N$ based on the added information of the preceding $N$ agents, and would pay up to the value of the information to defer. \\

\commentl{Flesh out Prior Information Timing Theory}

\subsection{Agent Strategy}
We will consider different types of agent strategies to model agent behavior. There are two types of agent strategies \mydef{myopic} $\Myopic$ and \mydef{farsighted} $\Farsighted$. Prediction markets are known to be myopically incentive compatible, which means that myopic agents bid truthfully. Similarly, \mydef{informed} $\Informed$ agents have an exogenous signal about the outcome whereas \mydef{uninformed} $\Uninformed$ agents can only base their decision on the market price. \\

$\AgentStrategy^{\Myopic\Informed}_\Agent(\Time)$ is an informed agent holding exogenous signal $\Agent_\AgentBelief$ who is willing to pay up to $\AgentBudget$ in order to move the market price $\OptionPrice_\Market$ as close as possible to their belief $\Agent_\AgentBelief$. \\

$\AgentStrategy^{\Myopic\Uninformed}_\Agent(\Time)$ is an uninformed agent holding no exogenous signal $\Agent_\AgentBelief$ who who is willing to pay up to $\AgentBudget$ in order to move the market price $\OptionPrice_\Market$ as close as possible to 1 if at time $\Time$ $\OptionPrice_\Market\Time \ge .5$ otherwise 0. \\

$\AgentStrategy^{\Farsighted\Informed}_\Agent(\Time)$ is an informed agent holding exogenous signal $\Agent_\AgentBelief$ who is attempting to maximize the expected value by bidding based on a linear combination of their signal $\Agent_\AgentBelief$ and the current market price $\OptionPrice_\Market$ accounting for how many agents $N$ have already bid. \\

\subsection{Problems}
  
  \begin{definition} (Market Maker Revenue)
  \label{def:mmr}
   Given a market maker $\OptionPrice_\Market$ and a set of participating agents
   $\Agents_\Option$, the revenue obtained from $\OptionPrice_\Market$ is defined as
     $$R(\OptionPrice_\Market, \Agents_\Option) = 
      \sum_{\Agent\in\Agents_\Option} \left[
	\int_{\Time = 0}^{\Time = \Option_\Time} 
	  \Price_\Market(\Option_0,\Agent,\Time, \AgentStrategy^{0}_{\Agent}(\Time))\AgentStrategy^{0}_{\Agent}(\Time)dt
	  + \int_{\Time = 0}^{\Time = \Option_\Time} 
	  \Price_\Market(\Option_1,\Agent,\Time, \AgentStrategy^{1}_{\Agent}(\Time))\AgentStrategy^{1}_{\Agent}(\Time)dt\right]$$
  \end{definition}
  
    \begin{definition} (Market Maker Cost)
  \label{def:mmc}
   Given a market maker $\OptionPrice_\Market$ and a set of participating agents
   $\Agents_\Option$, the cost to $\OptionPrice_\Market$ is defined as
     $$ C(\OptionPrice_\Market, \Agents_\Option) =
     \sum_{\Agent\in\Agents_\Option} \left[
	\int_{\Time = 0}^{\Time = \Option_\Time} 
	\Revealer(\Option_\Event)\AgentStrategy^{0}_{\Agent}(\Time)dt
	+\int_{\Time = 0}^{\Time = \Option_\Time} 
	\Revealer(\Option_\Event)\AgentStrategy^{1}_{\Agent}(\Time)dt
	\right]$$
  \end{definition}

      \begin{definition} (Market Maker Proft)
  \label{def:mmf}
   Given a market maker $\OptionPrice_\Market$ and a set of participating agents
   $\Agents_\Option$, the profit of $\OptionPrice_\Market$ is defined as
     $$ P(\OptionPrice_\Market, \Agents_\Option) = R(\OptionPrice_\Market, \Agents_\Option) 
     - C(\OptionPrice_\Market, \Agents_\Option)$$
  \end{definition}
  
\begin{definition} (Profit-Maximizing Market Maker).
\label{def:pmmm}
Among all Market Makers $\MarketMakers$, given a set of participating agents $\Agents_\Option$ find the one that maximizes Profit:
$$ PM(\MarketMakers, \Agents_\Option) = arg max_{\Agent \in \Agents_\Option} P(\OptionPrice_\Market, \Agents_\Option)$$
\end{definition}


